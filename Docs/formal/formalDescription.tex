%% LyX 2.0.5.1 created this file.  For more info, see http://www.lyx.org/.
%% Do not edit unless you really know what you are doing.
\documentclass[english]{article}
\usepackage[T1]{fontenc}
\usepackage[utf8]{luainputenc}
\usepackage{amssymb}

\makeatletter
%%%%%%%%%%%%%%%%%%%%%%%%%%%%%% User specified LaTeX commands.
\usepackage{bbm}


\usepackage{geometry}[0.5]

\usepackage{background}


%manage background
\SetBgOpacity{0.3}
\SetBgColor{black!20}

\makeatother

\usepackage{babel}
\begin{document}

\title{Daily urban mobility simulation using agent-based modeling\\
$\begin{array}{ccc}
\star &  & \star\\
 & \star
\end{array}$\\
Formal description of the model}
\maketitle
\begin{abstract}
We propose a simplified model of daily urban mobility that takes into
account individual choices between different types of transport. Feedback
from the positions of agents at micro-level gives integrated macro-flows
in the street network whose influenced themselves local speed and
sometimes route choices of agents. Flows have also a component representing
externalities such as traffic from outside the city that we can't
represent as agents. More than having an accurate model that we could
well calibrate on real data, the aim is to explore the aspect of influence
of decision-making rules, and that way of the type of population,
on the global patterns. An other application is to proceed to multi-criteria
optimisation on a modification of the public transport infrastructure.
We present here the formal description of our model.
\end{abstract}

\section{Global framework}

We work either on imaginary city or with projected geographic description
of a real city, so the space on which individuals move is a subset
$C\subset\mathbb{R}^{2}$. We suppose that the road network and public
transportation network are represented by two Euclidian graphs, with
$V=(x_{i},y_{i})_{1\leq i\leq n(G)}$ family of vertices, $G=(V,E)$
with $E\subset V\times V$, and with respectively the same notations
for $G'$. Furthermore, we assume the existence of an resistance function
$r(\cdot,t)$ on the edges of the road network, which will be anisotropic
in order to simplify the implementation, and that will depend on the
current charge of the edge. For the public transport, speeds in edges
only depend of time, and will be determined by real frequencies. For
use of street network by foot, we assume that no congestion is possible
so resistance is constant regarding time and charge. Edges also have
fixed capacity $c$.

The flow function on the edges $f(\cdot,t)$ will be used as an output
of the model and in the expression of the resistance function to have
feedback of traffic charge on agent speed.

Time is discretised given a variable step $\tau$, for which the value
is not supposed to influence the results under a certain precision
treshold, which determination will be one of the first aims of the
experiments.


\section{Agents description}

Agents can be of three different types, what has a direct consequence
in their time-table and decision-making.

The set of agents is described as $\mathcal{A}\simeq$ //ADD Formal
agent Description


\section{Movements in the city}

Given the time-table of an agent $a\in\mathcal{A}$, represented as
a function $T_{a}:D_{a}\rightarrow C$ with $D_{a}\in\mathcal{P}([0;24])$
(most probably defined on finite subset), 


\subsection{Decision-making}

//Add decision making tree.


\subsection{Travel by public transportation or foot}


\subsection{Heuristic for agent travel by car}

In order to get more realistic interactions between agents, we propose
to add to the feedback of flows on agents speed by scale-coupling
a local effect of roads charges on agent decision. Therefore, an agent
can at any time of his travel modify the route he has planned. The
structure of the heuristic we described here appears as a crucial
point in the modeling process since all choices were made according
to ``rational'' vision of human agents.

Each agent has for each ``daily'' travel some fixed paths (let say
arbitrary three) that he knows : $p=((p_{0},p_{1},p_{2})_{O_{i},D_{i}})_{i}$
with $O_{i},D_{i}\in C$. The agent $a$ executes the route step by
step by changing the position $(x_{a}(t+\tau),y_{a}(t+\tau))=m(t,p,a,r(e(a,t),t),c(e(a,t)),f(e(a,t),t))$
with $e(a,t)$ the next edge in the planned path at time t. 


\paragraph{Metric on paths in euclidian graph}

To justify the used heuristic for ``prefered path'' calculation,
we need a metric on path in graphs to justify the intuition that perturbing
the shortest path and recalculating the shortest path will lead to
a ``not so far'' path in the graph.

To define an exact continuity, we need a total metric : a good candidate
is $[d((a_{i})_{1\leq i\leq n},(a'_{j})_{1\leq j\leq n'})=\sum_{i=1}^{min(n,n')}(a_{i}-a'_{i})^{2}+\bbm{1}_{n=n'}\cdot +\infty +\bbm{1}_{a_{1}=a'_{1}&a_{min(n,n')}=a'_{min(n,n')}}\cdot +\infty]$
\[
\]

\end{document}
